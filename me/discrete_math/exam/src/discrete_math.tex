\documentclass{article}

\usepackage[utf8]{inputenc}
\usepackage[T1,T2A]{fontenc}
\usepackage{amsmath}
\usepackage{amsfonts}
\usepackage{amssymb}
\usepackage{mathtools}
\usepackage{bm}
\usepackage{esvect}
\usepackage{amsthm}
\usepackage{enumitem}
\usepackage[russian]{babel}

\newtheorem{theorem}{Теорема}
\newtheorem*{theorem*}{Теорема}

\theoremstyle{plain}
\newtheorem{example}{Пример}
\newtheorem*{example*}{Пример}

\theoremstyle{definition}
\newtheorem{definition}{Определение}[subsection]

\begin{document}
\sloppy

\title{Дискретная математика}
\date{\vspace{-5ex}}
\maketitle

\section{Множества}

\subsection{Понятие множества. Множества и подмножества. Способы задания множества}

Георг Кантор, основатель теории множеств, опредял \textit{множество} как «многое, мыслимое как единое целое». Можно дать и более строгую формулировку:

\begin{definition}
	\textit{Множество} — это математический объект, являющий совокупностью объектов произвольной природы, которые называются \textit{элементами} этого множества и обладают общим для них характеристическим свойством.
\end{definition}

\begin{definition}
	Множество \(S\) называется подмножеством множества \(M\), если \(\forall s \in S\; s \in M\).
\end{definition}

Существует несколько способов задания множества. Во-первых, можно задать множество просто перечислив все его элементы:

\[
	\mathbb{N} = \{1, 2, 3, \ldots\}.
\]

Также можно задать множество с помощью так называемого \textit{характеристического предиката}:

\[
	S = \{x \colon x \in \mathbb{Z} \land x > 30\}.
\]

Другим вариантом будет задание породжающей процедуры, например, для генерирования множества чисел последовательности Фибоначчи:

\[
	F = \{a \;\colon a_1 = 0, a_2 = 0, a_i = a_{i - 2} + a_{i - 1}, \;\text{где}\; i = 3, 4, \ldots\}.
\]

\subsection{Конечные и бесконечные множества. Сравнимые множества. Мощность множества. Булеан}

\begin{definition}
	Множество называется \textit{конечным}, если оно содержит конечное количество элементов. Пустое множество также считается конечным.
\end{definition}

Соответственно, множество, которое содержит бесконечное количество элементов называется бесконечным. Например, множество натуральных чисел \(\mathbb{N}\) является бесконечным.

Бесконечные множества делятся еще на вида: счетные и несчетные. Счетные множества — это те, элементы которых мы можем пронумеровать: первый, второй, третий, … Очевидно, что множество натуральных чисел является счетным. А что на счет множества действительных чисел? Можем ли мы пронумеровать все его элементы? Очевидно, нет. Поэтому множество действительных чисел является несчетным. А все множества с такой же мощностью как у множества действительных чисел называются континуальными.

Говоря формально, множество является счетным, если его можно взаимо-однозначно сопоставить со множеством натуральных чисел.

\begin{definition}
	Два множества \(A\) и \(B\) называют сравнимыми, если \(A \subset B\) или \(B \subset A\).
\end{definition}

Другими словами, два множества называются сравнимыми, если одно из них является подмножеством другого.

\begin{definition}
	\textit{Мощностью} множества или его \textit{кардинальным числом} называют количество элементов, которые содержатся в этом множестве. Мощность множества \(A\) обозначается через \(card(A)\),~\(\#A\)~или~\(|A|\).
\end{definition}

Рассмотрим множество \(\{1, 2, 3\}\). В нем три элемента, значит его мощность равна \(3\). А какова мощность множества натуральных чисел? Мощность множества натуральных чисел обозначается как \(\aleph_0\) (читается «алеф-ноль»). Мощность множества действительных чисел обозначается как \(\mathfrak{c}\). Предположение о том, что \(\mathfrak{c} = \aleph_0\) называется \textit{континуум-гипотезой}.

\begin{definition}
	\textit{Булеаном} \(2^A\) называют множество всех подмножеств данного множества \(A\).
\end{definition}

\begin{example*}
	Например, булеаном множества \(\{1, 2\}\) будет являтся множество \(\{ \{1, 2\}, \{1\}, \{2\}, \varnothing \}\).
\end{example*}

Очевидно, что количество элементов в булеане множества всегда больше количества элементов самого множества. Можно вывести и более общую формулу, выражающую зависимость между мощностью конечного множества и мощностью его булеана:

\[
	card(2^A) = 2^{card(A)}.
\]

\subsection{Парадоксы теории множеств}

\subsubsection{Парадокс Рассела}

Одним из парадоксов наивной теории множеств является парадокс Рассела. Будем называть множество «обычным», если оно не содержит себя в качестве своего элемента. И «необычным», если содержит. Допустим, у нас есть множество \(S\), содержащее абсолютно все «обычные» множества. Парадокс возникает, когда мы пытаемся понять, каким является множество \(S\) — обычным или необычным.

С одной стороны, если оно «обычное», то оно должно включать себя в качестве своего элемента, посколько состоит из всех «обычных» множеств. Но тогда оно не будет является «обычным», т. к. включает содержит само себя в качестве своего элемента.

С другой стороны, если оно «необычное», то должно влючать само себя в качестве своего элемента, т. к. это свойство всех «необычных» множеств. Однако оно не может включать себя в качестве своего элемента, т. к. состоит только из «обычных» множеств.

\subsubsection{Парадокс Кантора}

Пусть существует множество всех множеств \(S\). Тогда, по-определению, оно должно содержать свой булеан, т. е. \(2^S \subset S\). Очевидно, что если \(A \subset B\), то \(card(A) < card(B)\), но мощность булеана всегда больше мощности исходного множества. Получили противоречие.

\section{Отношения}

\subsection{Отношение. Бинарное отношение. Обратное отношение. Композиция бинарных отношений. Тождественное отношение. Универсальное отношение}

\begin{definition}
	Множество \(\varphi\) называют \textit{\(n\)-арным отношением} между элементами множеств \(A_1,\; A_2,\; \ldots,\; A_n\), если оно является подмножеством их декартова произведения \(A_1 \times A_2 \times \ldots \times A_n\).
\end{definition}

\begin{definition}
	\textit{Бинарным отношением} \(\varphi\) между элементами множеств \(A\) и \(B\) называют любое подмножество их декартова произведения. Другими словами, \(\varphi \subseteq A \times B\).
\end{definition}

\begin{definition}
	Пусть дано бинарное отношение \(\varphi \subseteq A \times B\). Тогда \textit{обратным бинарным отношением} будет называться отношение \(\varphi^{-1}\) такое, что

	\[
		\varphi^{-1} = \{(b, a) \;:\; (a, b) \in \varphi\}.
	\]
\end{definition}

\begin{definition}
	Пусть дано два бинарных отношения \(\rho \subseteq A \times B\) и \(\phi \subseteq B \times C\). Тогда их \textit{композицией} \(\rho \circ \phi\) будет называться бинарное отношение \(\mu \subseteq A \times C\):

	\[
		\mu = \rho \circ \phi = \{(a,\; c) \;\colon\; (\exists b \in B)\; [(a,\; b) \in \rho \;and\; (b,\; c) \in \phi]\}.
	\]
\end{definition}

\begin{definition}
	Бинарное отношение \(\varphi \subseteq A^2 = \{(a, a) \colon a \in A\}\) называется \textit{тождественным} и обозначается как \(\mathrm{id}_A\).
\end{definition}

\begin{definition}
	Отношение \(\varphi \subseteq A \times B = \{(a, b) \colon a \in A, b \in B\}\) называется \textit{универсальным}.
\end{definition}

\subsection{Степень отношения. Ядро отношения}

\begin{definition}
	\textit{Степенью} \(\rho^n\) бинарного отношения \(\rho\) называют композицию этого отношения с самим собой \(n\) раз: \[\rho\; \underbrace{\circ\; \ldots\; \circ}_\text{$n$ раз}\; \rho.\]
\end{definition}

\begin{definition}
	\textit{Ядром} отношения \(\rho\) называется композиция отношения и обратного ему: \(\rho \circ \rho^{-1}\).
\end{definition}

\subsection{Отношения эквивалентности. Теорема об отношении эквивалентности и разбиении множества. Классы эквивалентности. Фактор–множество}

\begin{definition}
	Бинарное отношение называется \textit{отношением эквивалентности}, если оно рефлексивно, симметрично и транзитивно.
\end{definition}

\begin{definition}
	Разбиением множества \(A\) называется множество его подмножеств такое, что

	\[
		\bigcup \limits _{{i \in I}}A_{i} = A \;\land\; [\forall i, j \colon i \neq j] (A_i \cap A_j = \varnothing).
	\]
\end{definition}

\begin{theorem*}
	Всякое отношение эквивалентности множества \(A\) определяет разбиение множества \(A\), причем среди элементов разбиения нет пустых. Верно и обратное: всякое разбиение множества \(A\), не содежащее пустых элементов, определяет отношении эквивалентности на множестве \(A\).
\end{theorem*}

\begin{definition}
	Пусть \(\equiv\) является отношением эквивалентности на множестве \(A\) и \(x \in A\). Тогда \textit{классом эквивалентности} для \(x\) называют подмножество элементов из \(A\), эквивалентных \(x\):

	\[
		[x]_{\equiv} = \{ y \colon y \in A \land y \equiv x \}.
	\]
\end{definition}

\begin{definition}
	Пусть \(R\) — отношение эквивалентности на множестве~\(A\). Тогда \textit{фактор–множеством} называют множество всех классов эквивалентности множества \(A\) по отношению \(R\) и обозначают как \(M / R\) или \(\{ [x] \}_{x \in A}\).
\end{definition}

\subsection{Замыкание бинарных отношений. Теорема о транзитивном замыкании}

Замкнутость множества относительно применения какой-либо операции означает, что многократное применение операции к элементам этого множества не выводит образующиеся в результате применения операции элементы за пределы исходного множества.

Например, множество натуральных чисел \(\mathbb{N}\) замкнуто относительно операции сложения, потому что при сложении любых двух натуральных чисел получается натуральное число. Однако \(\mathbb{N}\) не является замкнутым относительно операции деления, т. к. при делении двух натуральных чисел может получиться число, не являющееся натуральным.

\begin{definition}
	\(R^+ = \bigcup \limits ^{{\infty}} _{{i = 1}}R^{i}.\)
\end{definition}

\begin{definition}
	\textit{Транзитивное замыкание} отношения \(R\) на множестве \(M\) есть наименьшее транзитивное отношение на множестве \(M\), включающее \(R\).
\end{definition}

\begin{example*}
	Например, если элементами множества \(M\) являются люди, а отношение \(R \subset M^2\) — это отношение «является родителем», то транзитивным замыканием отношения \(R\) будет являться отношение «является предком».
\end{example*}

\begin{theorem*}
	\(R^+\) есть транзитивное замыкание~\(R\).
\end{theorem*}

\subsection{Функции. Инъекция, сюрьекция, биекция. Теорема о тотальной биекции}

\begin{definition}
	Функцией называют бинарное отношение, которое обладает свойством однозначности:

	\[
		(a, b) \in f \land (a, c) \in f \rightarrow b = c.
	\]
\end{definition}

Бинарное отношение \(f \subseteq X \times Y\), обладающее свойством однозначности называется \textit{функциональным} и записывается как \(f \colon X \rightarrow Y\).

\begin{definition}
	Функция называется \textit{инъективной}, если у каждого значения функции есть только один прообраз:

	\[
		y = f(x_1) \;and\; y = f(x_2) \rightarrow x_1 = x_2.
	\]
\end{definition}

\begin{definition}
	Функция \(f: X \rightarrow Y\) является \textit{сюръективной}, если областью ее значений является все множество \(Y\), т. е. если она принимает все возможные значения:

	\[
		\forall y \in Y \exists x \in X \colon (x, y) \in f.
	\]
\end{definition}

\begin{definition}
	Функция называется \textit{биективной}, если она сюръективна и инъективна одновременно.
\end{definition}

Биективная функция также называется взаимно-однозначным соответствием.

\section{Алгебраические структуры}

\subsection{Операции и их свойства. Алгебраическая структура. Модель. Примеры алгебраических структур}

\begin{definition}
	\textit{\(n\)-арной} (или \textit{\(n\)-местной}) \textit{операцией} на \(M\) называют всюду определенную на множестве \(M\) тотальную функцию от \(n\) аргументов.
\end{definition}

Если \(\varphi\) --- бинарная операция, т. е. \(\varphi \colon M^2 \rightarrow M\), то её обозначают как \(\varphi(a, b)\), где \(a, b \in M\). Иногда используют инфиксную форму записи \(a \circ b\), где \(\circ\) --- знак операции.

\begin{definition}
	Множество с набором определенных на нем операций \(\mathcal{A} = \langle M; \varphi_1, \ldots, \varphi_m \rangle\), где \(\varphi_i \colon M^{n_i} \rightarrow M\) называется \textit{алгебраической структурой}.
\end{definition}

\begin{definition}
	Алгебраическая структура с пустым множеством операций называется \textit{моделью}.
\end{definition}

\begin{example*}
	Одним из простейших примеров алгебраической структуры является множество целых чисел с операциями сложения и вычитания: \(\langle \mathcal{A}; +, -\rangle\).
\end{example*}

\subsection{Булева алгебра. Примеры булевых алгебр. Теорема Стоуна}

\begin{definition}
	\textit{Булевой алгеброй} называют алгебраическую систему \(\langle M; \land, \lor, \lnot, 0, 1\rangle\), причем для любых \(a, b, c \in M\) верны следующие аксиомы:

	\begin{enumerate}[label=\roman*.]
		\item \(a \land (b \land c) = (a \land b) \land c\) и \(a \lor (b \lor c) = (a \lor b) \lor c\) (\textit{ассоциативность});
		\item \(a \land b = b \land a\) и \(a \lor b = b \lor a\) (\textit{коммутативность});
		\item \(a \land (a \lor c) = a\) и \(a \lor (a \land b)\) (\textit{законы поглощения});
		\item \(a \land (b \lor c) = (a \land b) \lor (a \land c)\) и \(a \lor (b \land c) = (a \lor b) \land (a \lor c)\) (\textit{дистрибутивность});
		\item \(a \land \lnot a = 0\) и \(a \lor \lnot a = 1\) (\textit{дополнительность}).
	\end{enumerate}
\end{definition}

\begin{example*}
	В качестве примера булевой алгебры можно привести множество всех подмножеств данного множества \(M\), которое образует булеву алгебру относительно операций объединения, пересечения и унарной операции дополнения.
\end{example*}

\begin{theorem*}[Стоуна]
	Всякую булеву алгебру можно интерпретировать как булеву алгебру подмножеств некоторого множества.
\end{theorem*}

Другими словами, какой бы ни была булева алгебра мы можем считать её элементы подмножествами некоторого множества, а операции, соответственно,~--- теоретико-множественными операциями.

\subsection{Подалгебра. Теорема о непустом пересечении подалгебры}

\begin{definition}
	Если в алгебре \(\langle M; \Sigma \rangle\) мы рассмотрим \(X \subset M\) такое, что \(X\) замкнуто относительно всех операций из \(\Sigma\), то система \(\langle X; \Sigma_X \rangle\) образует \textit{подалгебру} алгебры \(\langle M; \Sigma \rangle\), где \(\Sigma_X\) состоит из сужений операций из \(\Sigma\) на \(X\).
\end{definition}

\begin{theorem*}[о не пустом пересечении подалгебр]
	Непустое пересечение подалгебр одной алгебры образует подалгебру той же алгебры.
\end{theorem*}

\subsection{Полугруппа. Примеры. Свободная полугруппа}

\begin{definition}
	\textit{Полугруппой} называют алгебраическую структуру с одной бинарной ассоциативной операций.
\end{definition}

\begin{example*}
	Например, алгебраическая система \(\langle A^+; \cdot \rangle\), где \(A^+\) — множество слов в алфавите, а \(\cdot\) — операция конкатенации строк, является полугруппой.
\end{example*}

\begin{definition}
	Если в полугруппе все различные слова, состоящие из образующих определяют различные элементы носителя, то полугруппа называется \textit{свободной}.
\end{definition}

\begin{example*}
	Например, системой образующих полугруппы \(\langle \mathbb{N}; + \rangle\) является множество \(\{1\}\). Так как различные слова в алфавите \(\{1\}\) — суть различные элементы носителя, то \(\langle \mathbb{N}; + \rangle\) является полугруппой.
\end{example*}

\subsection{Моноид. Примеры. Теорема о единственности единицы в моноиде}

\begin{definition}
	\textit{Моноидом} называют полугруппу \(\langle M; \circ, e\rangle\), в которой существует нейтральный элемент (также называемый единицей) \(e\) такой, что \(\forall m \in M \enspace m \circ e = e \circ m = m\).
\end{definition}

\begin{example*}
	Простейшими примерами моноидов являются \(\langle \mathbb{N}; +, 0\rangle\) и \(\langle \mathbb{R}; \cdot, 1 \rangle\).
\end{example*}

\begin{theorem*}
	Единица в моноиде единственна.
\end{theorem*}

\begin{proof}[Доказательство]
	Пойдем от противного и предположим, что существует два нейтральных элемента \(e_1\) и \(e_2\). Тогда \(e_1 = e_1 \circ e_2 = e_2 \circ e_1 = e_2\).
\end{proof}

\subsection{Группа. Примеры. Теорема о единственности обратного элемента в группе}

\begin{definition}
	\textit{Группой} называют моноид, в котором для каждого элемента существует элемент, обратный ему.
\end{definition}

\begin{example*}
	Примером группы является \(\langle \mathbb{Z}; + \rangle\), где для каждого целого числа существует обратное ему число с противоположным знаком.
\end{example*}

\begin{theorem*}
	Обратный элемент в группе единственнен.
\end{theorem*}

\begin{proof}[Доказательство]
	Пойдем от противного и предположим, что в группе \(\langle M; \circ \rangle\) для данного \(m \in M\) существует два обратных элемента \(a\) и \(b\). Тогда \(a = a \circ e = a \circ (m \circ b) = (a \circ m) \circ b = e \circ b = b\).
\end{proof}

\subsection{Теорема о свойствах операций в группе}

\begin{theorem*}
	В группе выполняются следующие соотношения:

	\begin{enumerate}
		\item \((a \circ b)^{-1} = b^{-1} \circ a^{-1}\);
		\item \(a \circ b = a \circ c \Rightarrow b = c\);
		\item \(b \circ a = c \circ a \Rightarrow b = c\);
		\item \((a^{-1})^{-1} = a\).  
	\end{enumerate}
\end{theorem*}

\begin{proof}[Доказательство]
	Каждое соотношение легко доказывается с помощью простых логических выводов:

	\begin{enumerate}
		\item \(
			(a \circ b)^{-1} = b^{-1} \circ a^{-1} \enspace \Rightarrow \enspace
			(a \circ b) \circ b^{-1} \circ a^{-1} {= e} \enspace \Rightarrow \enspace
			a \circ (b \circ b^{-1}) \circ a^{-1} {= e} \enspace \Rightarrow \enspace
			a \circ e \circ a^{-1} {= e} \enspace \Rightarrow \enspace
			a \circ a^{-1} {= e} \enspace \Rightarrow \enspace
			e = e
		\);

		\item \(
			a \circ b = a \circ c \enspace \Rightarrow \enspace
			a^{-1} \circ (a \circ b) = a^{-1} \circ (a \circ c) \enspace \Rightarrow \enspace
			(a^{-1} \circ a) \circ b = (a^{-1} \circ a) \circ c \enspace \Rightarrow \enspace
			e \circ b = e \circ c \enspace \Rightarrow \enspace
			b = c
		\);

		\item (\textit{доказывается по аналогии с предыдущим соотношением});

		\item (\textit{прямо следует из факта единственности обратного элемента в группе}).
	\end{enumerate}
\end{proof}

\subsection{Теорема об однозначности решения в группе уравнения \(a \times x = b\)}

\begin{theorem*}
	В группе можно однозначно решить уравнение \(a \times x = b\).
\end{theorem*}

\begin{proof}[Доказательство]
	\(a \times x = b \enspace \Rightarrow \enspace
	a^{-1} \times (a \times x) = a^{-1} \times b \enspace \Rightarrow \enspace
	(a^{-1} \times a) \times x = a^{-1} \times b \enspace \Rightarrow \enspace
	e \times x = a^{-1} \times b \enspace \Rightarrow \enspace
	x = a^{-1} \times b\).
\end{proof}

\subsection{Коммутативная группа. Примеры}

\begin{definition}
	\textit{Коммутативной} (или \textit{абелевой}) группой называют группу, в которой бинарная операция коммутативна.
\end{definition}

\begin{example*}
	Например, группа \(\langle \mathbb{Z}; +\rangle\) является абелевой группой, так как операция \(+\) обладает свойством коммутативности.
\end{example*}

\subsection{Кольцо. Примеры. Теорема о соотношениях в кольце}

\begin{definition}
	\textit{Кольцом} называют алгебраическую систему \(\langle M; +, \times\rangle\), являющуюся абелевой группой по сложению, полугруппой по умножению, и обладающая двухсторонней дистрибутивностью умножения относительно сложения.
\end{definition}

\begin{example*}
	Простейшим примером кольца является множество целых чисел с обычными операциями сложения и умножения.
\end{example*}

\begin{theorem*}
	В кольце выполняются следующие соотношения:

	\begin{enumerate}
		\item \(0 \cdot a = a \cdot 0 = 0\);
		\item \(a \cdot (-b) = (-a) \cdot b = -(a \cdot b)\);
		\item \((-a) \cdot (-b) = a \cdot b\).
	\end{enumerate}
\end{theorem*}

\begin{proof}[Доказательство]
	Все соотношения доказываются простыми логическими цепочками:

	\begin{enumerate}
		\item \(a \cdot 0 = a \cdot (x - x) = ax - ax = 0\), а также \(0 \cdot a = (x - x) \cdot a = xa - xa = 0\); 
		\item \(ab + (-a)b = [a + (-a)]b = 0 \cdot b = 0\), а также \(ab + a(-b) = a[b + (-b)] = a \cdot 0 = 0\);
		\item \((-a)(-b) = -[a(-b)] = -(-ab) = ab\).
	\end{enumerate}
\end{proof}

\section{Булевы функции}

\subsection{Булевы функции. Элементарные булевы функции. Способы задания булевых функций}

\begin{definition}
	Отображение \(B^n \rightarrow B\), где \(B = \{0, 1\}\) называется \textit{булевой функцией от \(n\)-переменных}.
\end{definition}

При \(n\) равным нулю количество булевых функций сводится к двум. Первая из них тождественно равна 0, а вторая — 1. Их называют \textit{булевыми константами} — тождественный ноль и тождественная единица.

Всякая булева функция задается конечным набором значений, что позволяет представить ее в виде, например, \textit{таблицы истинности}.

Множество всех булевых функций обозначается как \(P_2\), а множество всех булевых функций от \(n\) переменных как \(P_2(n)\).

\subsection{Существенная и несущественная переменные. Теорема о числе булевых функций, зависящих от \(n\) переменных}

\begin{definition}
	Есть для любых двух булевых векторов, отличающихся лишь в значении этой переменной, значение функции на них совпадает, то такая переменная называется \textit{несущественной} или \textit{фиктивной}.
\end{definition}

Другими словами, переменная является фиктивной, если значение функции не зависит от значения этой переменной.

\begin{theorem*}
	Число булевых функций, зависящих от \(n\) переменных равно \(2^{2^n}\).
\end{theorem*}

\begin{proof}[Доказательство]
	Количество булевых векторов длины \(n\) равно \(2^n\). Поскольку на каждом из булевых векторов функция может принимать значение либо \(0\), либо \(1\), то количество всех булевых функций от \(n\) переменных равно \(2^{2^n}\).
\end{proof}

\subsection{Формулы. Интерпретация формул. Равносильные формулы. Правила эквивалентных преобразований формул}

\begin{definition}
	Пусть даны две функции \(f(x_1, \ldots, x_n)\) и \(g(y_1, \ldots, y_m)\) тогда \textit{подстановкой} функции \(g\) в функцию \(f\) называется замена \(i\)-того аргумента функции \(f\) значением функции \(g\):
	\[h(x_1, \ldots, x_{n+m-1}) = f(x_1, \ldots, x_{i-1}, g(x_i, \ldots, x_{i+m-1}), x_{i+m}, \ldots, x_{n+m-1}),\]

	а сама функция \(h\) называется \textit{суперпозицией} функций \(f\) и \(g\).
\end{definition}

\begin{definition}
	Пусть \(\Sigma \subset P_2\).

	\begin{enumerate}
		\item Каждая функция \(\varphi(x_1, \ldots, x_n) \in \Sigma\) является \textit{формулой} над \(\Sigma\).
		\item Рассмотрим функцию \(\varphi(x_1, \ldots, x_n)\) и выражения \(A_1, \ldots, A_n\), где \(A_i\) --- выражение, являющееся либо формулой над \(\Sigma\), либо символом переменной из множества \(U\). Тогда выражение \(\varphi(A_1, \ldots, A_n)\) назыается \textit{формулой} над \(\Sigma\).
	\end{enumerate}
\end{definition}

Сопоставим теперь каждой формуле над \(\Sigma\) функцию из \(P_2\):

\begin{enumerate}
	\item Если \(F = f(x_1, \ldots, x_n)\), где \(F\) есть формула над \(Sigma\), то формуле \(F\) сопоставим функцию \(f(x_1, \ldots, x_n)\).
	\item Пусть \(F = f(A_1, \ldots, A_n)\), где \(A_i\) --- это выражение, представляющее либо формулу над \(\Sigma\), либо символ переменной из множества \(U\). Тогда согласно предположению индукции каждому выражению \(A_i\) сопоставлена либо функция \(f_i \in P_2\), либо тождественная функция \(f_i = x_s\). Тогда формуле \(F = f(A_1, \ldots, A_n)\) соответствует функция \(f(f_0, \ldots, f_n)\).
\end{enumerate}

\begin{definition}
	Две формулы называются \textit{равносильными} (\textit{эквивалентными}), если им соответствуют равные функции.
\end{definition}

\subsection{Алгебра булевых функций. Теорема об алгебре булевых функций}

\begin{definition}
	Алгебра \(\langle P_2; \land, \lor, \lnot \rangle\) называется \textit{алгеброй булевых функций}.
\end{definition}

\subsection{Функция двойственная к данной. Терема о принципе двойственности}

\begin{definition}
	Функция \(g(x_1, \ldots, x_n)\) является \textit{двойственной} функции \(f(x_1, \ldots, x_n)\), если

	\[
		f(\overline{x_1}, \ldots, \overline{x_n}) = \overline{g(x_1, \ldots, x_n)}.
	\]
\end{definition}

\begin{theorem*}[о принципе двойственности]
	Если в формуле \(F\), представляющей булевую функцию \(\varphi(x_1, \ldots, x_n)\), все знаки функций заменить на соответствуюшие им знаки двойственных функций, то полученная формула \(F^*\) будет представлять функцию \(\varphi^*\), двойственную исходной.
\end{theorem*}

\subsection{Алгебра Жигалкина. Полином Жигалкина. Теорема о полиноме Жигалкина}

\begin{definition}
	Алгебра \(\langle P_2; \&, \oplus \rangle\) называется \textit{алгеброй Жигалкина}.
\end{definition}

\begin{definition}
	Полином вида
	\[a \oplus a_1X_1 \oplus a_2X_2 \oplus \dots \oplus a_nX_n \oplus a_{12}X_1X_2 \oplus \dots \oplus a_{1\ldots n}X_1\ldots X_n,\]
	\[a, \ldots, a_{1\ldots n} \in \{0, 1\}\]

	называют \textit{полиномом Жигалкина}.
\end{definition}

\begin{theorem*}
	Любая булева функция может быть единственным образом представлена в виде полинома Жигалкина.
\end{theorem*}

\begin{proof}[Доказательство]
	Количество всех булевых функций от \(n\) переменных равно \(2^{2^n}\). Количество различных слагаемых полинома Жигалкина от \(n\) переменных равно количеству различных подмножеств множества из \(n\) элементов, то есть \(2^n\). Количество различных полиномов, которые можно образовать из этих слагаемых равно \(2^{2^n}\). Таким образом, количество всех булевых функций от \(n\) переменных равно количеству всех различных полиномов Жигалкина от \(n\) переменных.

	Так как разным функциям соответсвует разные полиномы (одна и та же формула не может представлять разные фунции), то между множеством всех булевых функций от \(n\) перменных и множество всех полиномов Жигалкина от \(n\) переменных установлено взаимно-однозначное соответствие. 
\end{proof}

\section{Кодирование}

\subsection{Кодирование. Функция кодирования. Декодирование}

\begin{definition}
	Пусть задан алфавит \(\mathfrak{A} = \{a_1, \ldots, a_n\}\) состоящий из конечного числа букв. Конечную последовательность символов из \(\mathfrak{A}\)

	\[
		A = a_{i_1}a_{i_2}\ldots a_{i_n}
	\]

	будем называть \textit{словом} в алфавите \(\mathfrak{A}\).
\end{definition}

\begin{definition}
	Пусть \(S = S(\mathfrak{A})\) --- множество всех непустых слов в алфавите \(\mathfrak{A}\), а \(S' \subset S\). Слова из \(S'\) называются \textit{сообщениями}.
\end{definition}

\begin{definition}
	Пусть дан алфавит \(\mathfrak{B} = \{b_1, \ldots, b_q\}\). Через \(B\) обозначим слово в алфавите \(\mathfrak{B}\), через \(S(\mathfrak{B})\) --- множество всех непустых слов в алфавите \(\mathfrak{B}\). Пусть дана функция \[F \colon S'(\mathfrak{A}) \rightarrow S(\mathfrak{B}),\] которая каждому слову \(A \in S'(\mathfrak{A})\) ставит в соответствие слово \(B = F(A), B \in S(\mathfrak{B})\).

	Слово \(B\) будем называть \textit{кодом сообщения \(A\)}, а переход от слова \(A\) к его коду --- \textit{кодированием}.
\end{definition}

\begin{definition}
	Обратная функция \(F^{-1}\) (если она существует) называется \textit{функцией декодирования}.
\end{definition}

\subsection{Алфавитное кодирование. Схема кодирования. Равномерное кодирование}

\begin{definition}
	Рассмотрим соответствие между символами алфавита \(\mathfrak{A}\) и некоторыми словами в алфавите \(\mathfrak{B}\): \[a_i \mapsto B_i.\] Это соответствие называют \textit{схемой кодирования}. Схема определяет \textit{алфавитное кодирование} следующим образом: каждому слову \(A = a_1 \ldots a_n\) из \(S(\mathfrak{A})\) ставится в соответствие слово \(B = B_{i_1}\ldots B_{i_n}\), называемое \textit{кодом} слова \(A\). Слова \(B_{i_1}, \ldots, B_{i_n}\) называют \textit{элементарными кодами}. 
\end{definition}

Примером алфавитного кодирования может служить \textit{азбука Морзе}, в которой каждой букве латинского алфавита ставится в соответствие последовательность точек и тире. Другим примером является \textit{двоично-десятичное кодирование}.

\begin{definition}
	Если длины всех элементарных кодов равны, то кодирование называют \textit{равномерным}.
\end{definition}

\subsection{Свойство префиксности схемы кодирования. Взаимно-однозначное кодирование. Теорема-условие однозначности декодирования}

\begin{definition}
	Допустим, \(S'(\mathfrak{A}) = S(\mathfrak{A})\), т. е. источник сообщений порождает множество всех слова алфавита \(\mathfrak{A}\). Если отображение множества всех слов исходного алфавита на множество всех кодов взаимно-однозначно, то такое кодирование называется \textit{взаимно-однозначным}.
\end{definition}

Возникает вопрос: можно ли по схеме \(\Sigma\) понять, обладает ли кодирование свойством взаимной однозначности. Трудность решения состоит в том, что для непосредственной проверки взаимной однозначности необходимо проверить бесконечное количество слов. Однако простым достаточным признаком взаимо-однозначности кодирования является \textit{условие префиксности}.

\begin{definition}
	Пусть слово \(B\) имеет вид \(B = B_1B_2\). Тогда \(B_1\) называется \textit{началом} или \textit{префиксом} слова \(B\), а \(B_2\), соответственно, --- \textit{концом} или \textit{окончанием} слова \(B\).
\end{definition}

\begin{definition}
	Схема \(\Sigma\) обладает \textit{свойством префикса}, если для любых \(i\) и \(j\) \((1 \leq i, j \leq r, i \neq j)\) слово \(B_i\) не является префиксом слова \(B_j\).
\end{definition}

\begin{theorem*}[об однозначности декодирования]
	Если схема \(\Sigma\) обладает свойством префикса, то алфавитное кодирование будет взаимно-однозначным.
\end{theorem*}

\end{document}
